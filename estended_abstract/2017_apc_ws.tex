%%%%%%%%%%%%%%%%%%%%%%%%%%%%%%%%%%%%%%%%%%%%%%%%%%%%%%%%%%%%%%%%%%%%%%%%%%%%%%%%
%2345678901234567890123456789012345678901234567890123456789012345678901234567890
%        1         2         3         4         5         6         7         8

\documentclass[letterpaper, 10pt, conference]{ieeeconf}                             
\IEEEoverridecommandlockouts         % Needed if you want to use the \thanks command
                                                     
\overrideIEEEmargins                                                            


\usepackage{times} % assumes new font selection scheme installed
\usepackage{amsmath}
\usepackage{amssymb,longtable,calc}
\usepackage{mathptmx}
\usepackage[T1]{fontenc}                                                        
\usepackage[utf8]{inputenc}                                                     
\usepackage[english]{babel}                                                     
\usepackage{epsfig}                                                             
\usepackage{subfigure}                                                          
\usepackage{textcomp} %<- allows to use \textdegree but may overwrite           
                      %other settings                                           
\usepackage[textwidth=2cm,colorinlistoftodos,disable]{todonotes} %add disable   
     
										                         
\usepackage{tikz}                                                               
\usetikzlibrary{arrows,positioning,fit,shapes,calc}
\usetikzlibrary{matrix}
\usepackage{flushend}                                                           
\usepackage{hyperref}  
\usepackage{amsmath}                                                         
% \usepackage{multirow}   

\usepackage{algorithm}    
\usepackage{algorithmic}

\usepackage{pgfplots} 
\usepackage{pgfplotstable}

\usepackage{cite}

% helper packages to work on the draft
\usepackage[tikz]{bclogo}
\usepackage{lipsum}

\usepackage{standalone}

\pgfplotsset{compat=newest}
\pgfplotsset{ 
  tick label style={font=\footnotesize}, 
  label style={font=\footnotesize}, 
  legend style={font=\footnotesize},
  title style = {font=\small}
}
\pgfplotscreateplotcyclelist{line style}{% 
  solid, mark options = {scale = .75}, every mark/.append style={fill=gray},mark=*\\% 
  densely dashed,mark options = {scale = .75},every mark/.append style={solid,fill=gray},mark=*\\% 
  densely dotted,mark options = {scale = .75},every mark/.append style={solid,fill=gray},mark=*\\% 
  dashed,mark options = {scale = .75},every mark/.append style={solid,fill=gray},mark=*\\% 
  dotted,mark options = {scale = .75},every mark/.append style={solid,fill=gray},mark=*\\% 
}
\pgfplotscreateplotcyclelist{bar style}{% 
  solid, fill=black!60!white\\%
  solid, fill=black!45!white\\%
  solid, fill=black!35!white\\%
  solid, fill=black!25!white\\%
}


\usepackage{xspace}
\makeatletter                                                                   
\DeclareTextCommandDefault{\textregisteredalt}{\footnotesize\textcircled{%      
      \check@mathfonts\fontsize\sf@size\z@\math@fontsfalse\selectfont R}}       
\DeclareRobustCommand\onedot{\futurelet\@let@token\@onedot}                     
\def\@onedot{\ifx\@let@token.\else.\null\fi\xspace}                             
\def\eg{e.g\onedot}                                                             
\def\ie{i.e\onedot}                                                             
\def\vgl{see }                                                                  
\def\Fig{Fig\onedot }                                                           
\def\Tab{Tab\onedot }                                                           
\def\Eq{Eq\onedot }
\def\Sec{Sec\onedot}                                                            
\def\etc{etc\onedot}                                                            
\def\etal{\textsl{et al}\onedot}                                                
\def\argmin{\mathop{\rm arg\,min}}                                              
\makeatother
                                                                    
\definecolor{lightGray}{rgb}{0.0,0.0,0.0}
                                                                                
\title{\LARGE \bf The Importance of Being Bryce}                                                                               
                                                                                
                                                                                
\author{Hang Kayu ~~ Francisco Vina ~~ Michele Colendachise ~~ Karl Pauwels ~~ Alessandro Pieropan ~~ Danica Kragic%
\thanks{}%
\thanks{}
\thanks{}}

\begin{document}                                                                
                                                                                
\maketitle                                                                      
\thispagestyle{empty}                                                           
\pagestyle{empty}



%%%%%%%%%%%%%%%%%%%%%%%%%%%%%% ABSTRACT %%%%%%%%%%%%%%%%%%%%%%%%%%%%%%%%%%%%%
\begin{abstract}

In this paper we will present our framework used in the Amazon Picking Challenge in 2015 and some lessons-learned that may prove useful to researchers and future teams participating in the competition. The competition proved to be a very useful occasion to integrate the work of various researchers at the Robotics, Perception and Learning laboratory of KTH, measure the performance of our robotics system and define the future direction of our research.

\end{abstract}

%%%%%%%%%%%%%%%%%%%%%%%%%%%%%% INTRODUCTION %%%%%%%%%%%%%%%%%%%%%%%%%%%%%%%%%%%
% \begin{}
\section{INTRODUCTION}
\label{sec:introduction}

There are three main criteria engineers evaluates when determining the need of robots in certain applications: dirty, dull and dangerous. Those are known as the 3D of Robotics. The application proposed by the Amazon Picking Challenge meets certainly the second criteria as picking and placing objects in boxes could be a very repetitive and boring job. However, despite the defined and controlled environment the application of robots is still very challenging due to the nature of the objects to handle.
In this work we present the framework we develop at the Robotic, Perception and Learning lab (RPL) in 2015 with the purpose of sharing the lessons learned with the community.
First we will describe the platform used in the competition in Sec.\ref{sec:platform} to motivate the strategy we adopted in Sec.\ref{sec:strategy}. Then we will describe the core of our system that controls the whole pipeline of actions using behavioural trees in Sec.\ref{sec:trees}. Then we will describe our perception module starting with the localization of the shelf in Sec.\ref{sec:shelf} and detections of the objects \ref{sec:vision}. Finally we will describe our grasping strategy in Sec.~\ref{sec:grasping} and draw some conclusion about the limitation of our system in Sec.\ref{sec:conclusion}.

%% %%%%%%%%%%%%%%%%%%%%%%%%%%% PR2 %%%%%%%%%%%%%%%%%%%%%%%%%%%%%%%% 
\section{Platform}
\label{sec:platform}



%% %%%%%%%%%%%%%%%%%%%%%%%%%%% PR2 %%%%%%%%%%%%%%%%%%%%%%%%%%%%%%%% 
\section{Strategy}
\label{sec:strategy}

In 2015 the competition consisted in picking one object from each of the 12 bins of the shelf within 20 minutes. Each bin could contain from 1 up to 4 objects making the recognition and grasping of objects increasingly difficult. In order to design our strategy we had 3 main limitations to consider. First the PR2 could not reach the highest level of the shelf where 3 bin were located. Second two of the objects of the competition were bigger than the maximum aperture of the PR2 gripper. Third, it takes the PR2 about 30 seconds to raise the torso from the lowest position to the highest. Given the time requirements that operation resulted very expensive. Therefore our strategy consisted in starting from the lowest level of the shelf giving priority to the bins with one or two objects. Once the bins were cleared the PR2 would raise the torso to address the next level of the shelf. The more complex bins with multiple objects were left at the end disregarding the level they were located.

%% %%%%%%%%%%%%%%%%%%%%%%%%%%% BTS %%%%%%%%%%%%%%%%%%%%%%%%%%%%%%%%
\section{Behavior Trees}
\label{sec:trees}


BTs are a graphical mathematical model for reactive fault tolerant action executions. They were introduced in the video-game industry to control non-player characters, and they are now an established tool appearing in textbooks \cite{millington2009artificial,rabin2014gameAiPro} and generic game-coding software such as Pygame1, Craft AI 2, and the Unreal Engine3. In robotics, BTs are appreciated for being highly modular, flexible and reusable, and have also been shown to generalize other successful control architectures such as the Subsumption architecture, Decision Trees~\cite{colledachise17tro} and the Teleo-reactive Paradigm~\cite{Colledanchise16iros}.
%\subsection{Semantic}
%Here we breafly describe the semantic of BTs. An exaustive description can be found in~\cite{colledachise17tro}.
% 
%A BT is a directed rooted tree with the the common definition of \emph{parent} and \emph{child} node. Graphically, the children of nodes are placed below it. The children nodes are executed in the order from left to right.
%
%The execution of a BT begins from the root node that sends \emph{ticks}~\footnote{A tick is a signal that allows the execution of a node} with a given frequency to its (only) child. When a parent sends a tick to a child, the execution of this is allowed. The child returns to the parent a status \emph{running} if its execution has not finished yet, \emph{success} if it has achieved its goal, or \emph{failure} otherwise.\\ 
%There are four types of internal nodes (fallback, sequence, parallel, and decorator) and two types of leaf nodes (action and condition). Below we describe the execution of the nodes used in our framework.
%
%\paragraph*{Fallback}
%The fallback node send ticks to its children from the  left, returning success (running) when it finds a child that returns success (running). It returns failure only when all the children return failure. When a child returns running or success, the fallback node does not send ticks the next child (if any).
%A fallback node is graphically represented by a box labeled with a \say{?}, as in Figure~\ref{bg.fig.sel}.
%
%\paragraph*{Sequence}
%The sequence node sends ticks to its children from the  left, returning failure (running) when it finds a child that returns failure (running). It returns success only when all the children return success. When a child returns running or failure, the sequence node does not send ticks the next child (if any). A sequence node is graphically represented by a box labeled with a \say{$\rightarrow$}, as in Figure~\ref{bg.fig.seq}.
%
%\paragraph*{Action}
%The action node performs an action. It return running while the action is being performed. It return success of the action is completed correctly, otherwise it return failure. An action node is graphically represented by a rectangle labeled with the name of the action, as in Figure~\ref{bg.fig.seq}.
%
%
%\paragraph*{Condition}
%The condition node checks if a condition is satisfied or not, returning success or failure accordingly.  An action node is graphically represented by an ellipse labeled with the name of the condition, as in Figure~\ref{bg.fig.seq}.
%
%
%\subsection{BTs in APC}
In our framework, the use of BTs allowed us to have a control architecture that is:

\begin{itemize}
\item \textbf{Reactive:} The Reactive execution allowed us to have a system that rapidly reacts to unexpected changes. For example, if an object slips out of the robot gripper, the robot will automatically stop and pick it up again without the need to re-plan or change the BT; or if the position of a robot is lost, the robot will re-execute the routine of the object detection.
\item \textbf{Modular:} The Modular design allowed us to subdivide the behavior into smaller modules, that were independently created and then used in different stages of the project. This design allowed our heterogeneity developers’ expertise, letting developers implementing their code in their preferred programming language and style.
\item \textbf{Fault Tolerant:} The fault tolerant allowed us to handle actions failure by composing different actions meant for the same goals in a fallback. (e.g. different types of grasps).
\end{itemize}





%% %%%%%%%%%%%%%%%%%%%%%%%%%%% SHELF %%%%%%%%%%%%%%%%%%%%%%%%%%%%%%%% 
\section{Shelf}
\label{sec:shelf}



%% %%%%%%%%%%%%%%%%%%%%%%%%%%% VISION %%%%%%%%%%%%%%%%%%%%%%%%%%%%%%%%
\section{Vision}
\label{sec:vision}



%% %%%%%%%%%%%%%%%%%%%%%%%%%%% GRASPING %%%%%%%%%%%%%%%%%%%%%%%%%%%%%%%%
\section{Grasping}
\label{sec:grasping}



\section{Conclusion}
\label{sec:conclusion}

We presented the framework used in the competition in 2015 and all the challenges we had to face. We hope that our work can be useful to future teams 

\end{document}
